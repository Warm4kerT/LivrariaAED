
\begin{DoxyEnumerate}
\item Algoritmo Número 1\+: 
\begin{DoxyCode}{0}
\DoxyCodeLine{ListaCompra vendasNumPeriodo(int mes, int ano)\{}
\DoxyCodeLine{    Lista P = mainLista;}
\DoxyCodeLine{    ListaCompra final = NULL;}
\DoxyCodeLine{    ListaCompra C = NULL;}
\DoxyCodeLine{}
\DoxyCodeLine{    while(P!=NULL)\{}
\DoxyCodeLine{        C = P-\/>Cli.lista;}
\DoxyCodeLine{        while(C!=NULL)\{}
\DoxyCodeLine{            if((C-\/>dataVenda.mes == mes) \&\& (C-\/>dataVenda.ano == ano))\{}
\DoxyCodeLine{                final = InserirInicioCompra(C-\/>ISBN, C-\/>quantidade, C-\/>precoTotal, C-\/>dataVenda, final);}
\DoxyCodeLine{            \}}
\DoxyCodeLine{            C = C-\/>Prox;}
\DoxyCodeLine{        \}}
\DoxyCodeLine{        P = P-\/>Prox;}
\DoxyCodeLine{    \}}
\DoxyCodeLine{    }
\DoxyCodeLine{    return final;}
\DoxyCodeLine{\}}
\end{DoxyCode}
 Explicação\+: Função de procura de compras num dado mês e ano. A função vai iterando na lista de Clientes e verifica cada lista de compras que o cliente contem. Se essa compra pertencer a esse mês e ano então adiciona a uma lista de Compras e no final devolve-\/a.
\item Algoritmo Número 2\+: 
\begin{DoxyCode}{0}
\DoxyCodeLine{void ultimaVendaLivro(int ISBN)\{}
\DoxyCodeLine{    Lista P = mainLista;}
\DoxyCodeLine{    ListaCompra C = NULL;}
\DoxyCodeLine{    Data final = newData(0,0,0);}
\DoxyCodeLine{    Data aux;}
\DoxyCodeLine{}
\DoxyCodeLine{    while(P!=NULL)\{}
\DoxyCodeLine{        C = P-\/>Cli.lista;}
\DoxyCodeLine{        while(C!=NULL)\{}
\DoxyCodeLine{            if(C-\/>ISBN == ISBN)\{}
\DoxyCodeLine{                aux = C-\/>dataVenda;}
\DoxyCodeLine{}
\DoxyCodeLine{                if( aux.ano > final.ano )\{}
\DoxyCodeLine{                    final = aux;}
\DoxyCodeLine{                \}else if( aux.ano == final.ano)\{}
\DoxyCodeLine{                    if( aux.mes > final.mes )\{}
\DoxyCodeLine{                        final = aux;}
\DoxyCodeLine{                    \}else if( aux.mes == final.mes )\{}
\DoxyCodeLine{                        if( aux.dia > final.dia )\{}
\DoxyCodeLine{                            final = aux;}
\DoxyCodeLine{                        \}else if( aux.dia == final.dia )\{}
\DoxyCodeLine{                            final = aux;}
\DoxyCodeLine{                        \}}
\DoxyCodeLine{                    \}}
\DoxyCodeLine{                \}      }
\DoxyCodeLine{            \}}
\DoxyCodeLine{            C = C-\/>Prox;}
\DoxyCodeLine{        \} }
\DoxyCodeLine{        P = P-\/>Prox;      }
\DoxyCodeLine{    \}}
\DoxyCodeLine{}
\DoxyCodeLine{    if(final.ano == 0 || final.dia == 0 || final.mes == 0)\{}
\DoxyCodeLine{        printf("Nunca foi vendido!\(\backslash\)n\(\backslash\)n");}
\DoxyCodeLine{    \}else\{}
\DoxyCodeLine{        printf("Ultima compra do Livro \%d : \%d/\%d/\%d\(\backslash\)n\(\backslash\)n",ISBN,final.dia,final.mes,final.ano);}
\DoxyCodeLine{    \}      }
\DoxyCodeLine{\}}
\end{DoxyCode}
 Explicação\+: Esta função procura a mais recente compra de um \mbox{\hyperlink{structLivro}{Livro}} especificado pelo utilizador. Tal como a função alterior procura em cada lista de Compras de cada \mbox{\hyperlink{structCliente}{Cliente}}. Se encontrar alguma compra com o I\+S\+BN dado pelo Utilizador então guarda essa \mbox{\hyperlink{structData}{Data}}, até encontrar uma mais recente com esses critérios. No final apresenta a data.
\item Algoritmo Número 3\+: 
\begin{DoxyCode}{0}
\DoxyCodeLine{void quantidadeVendidaCliente(int NIF)\{}
\DoxyCodeLine{    Lista search = PesquisaPorm(mainLista,NIF);}
\DoxyCodeLine{}
\DoxyCodeLine{    if(search == NULL)\{}
\DoxyCodeLine{        printf("Cliente não encontrado\(\backslash\)n\(\backslash\)n");}
\DoxyCodeLine{        return;}
\DoxyCodeLine{    \}}
\DoxyCodeLine{}
\DoxyCodeLine{    ListaCompra C = search-\/>Cli.lista;}
\DoxyCodeLine{    int total = 0;}
\DoxyCodeLine{}
\DoxyCodeLine{    while(C!=NULL)\{}
\DoxyCodeLine{        total = total + C-\/>quantidade;}
\DoxyCodeLine{        C = C-\/>Prox;}
\DoxyCodeLine{    \}}
\DoxyCodeLine{}
\DoxyCodeLine{    printf("Quantidade vendida ao Cliente \%d : \%d\(\backslash\)n\(\backslash\)n",NIF,total);}
\DoxyCodeLine{}
\DoxyCodeLine{\}}
\end{DoxyCode}
 Explicação\+: Função simples de interpeção, conta o numero de Livros que um \mbox{\hyperlink{structCliente}{Cliente}} comprou. É invocada um função de pesquisa do \mbox{\hyperlink{structCliente}{Cliente}} dado pelo Utilizador e esse \mbox{\hyperlink{structCliente}{Cliente}} e guardado num apontador. Depois iteramos na lista de Compras contando o numero de livros. No final esse valor é apresentado.
\item Algoritmo Número 4\+: 
\begin{DoxyCode}{0}
\DoxyCodeLine{Tree LivrosMaisVendidosK(ListaCompra L, int k)\{}
\DoxyCodeLine{    ListaCompra P = L;}
\DoxyCodeLine{    ListaCompra T = NULL, searchL = NULL;}
\DoxyCodeLine{    Tree final, searchT;}
\DoxyCodeLine{}
\DoxyCodeLine{    while(P!=NULL)\{}
\DoxyCodeLine{        if(PesquisaCompra(P-\/>ISBN,T)==1)\{}
\DoxyCodeLine{            searchL = ProcurarAnteriorCompraISBN(P-\/>ISBN,T)-\/>Prox;}
\DoxyCodeLine{            searchL-\/>quantidade = searchL-\/>quantidade + P-\/>quantidade;}
\DoxyCodeLine{        \}else\{}
\DoxyCodeLine{            T = InserirInicioCompra(P-\/>ISBN, P-\/>quantidade, P-\/>precoTotal, P-\/>dataVenda,T);}
\DoxyCodeLine{        \}}
\DoxyCodeLine{}
\DoxyCodeLine{        P = P-\/>Prox;}
\DoxyCodeLine{    \}}
\DoxyCodeLine{}
\DoxyCodeLine{    T = bubbleSortCompra(T);}
\DoxyCodeLine{}
\DoxyCodeLine{    for(int i = k+1; i>=0; i-\/-\/)\{}
\DoxyCodeLine{        searchT = searchTreeISBN(mainTree,T-\/>ISBN);}
\DoxyCodeLine{        final = addNodoTree(final,searchT-\/>book);}
\DoxyCodeLine{        if(T-\/>Prox == NULL)\{}
\DoxyCodeLine{            break;}
\DoxyCodeLine{        \}}
\DoxyCodeLine{        T = T-\/>Prox;}
\DoxyCodeLine{    \}}
\DoxyCodeLine{}
\DoxyCodeLine{    return final;}
\DoxyCodeLine{\}}
\end{DoxyCode}
 Explicação\+: Nesta função recebemos já uma lista de Compras do mês e ano especificos e depois fazemos uma \char`\"{}condensação\char`\"{} dessa lista. Isto é removemos compras dos mesmos Livros guardando numa so compra a quantidade total de Livros vendidos numa so compra. Após esse processo organizamos a lista através de um Bubble\+Sort e apresentamos k Livros através do I\+S\+BN da \mbox{\hyperlink{structCompra}{Compra}}.
\item Algoritmo Número 5\+: 
\begin{DoxyCode}{0}
\DoxyCodeLine{int wastedLoopTree(Tree T)\{}
\DoxyCodeLine{}
\DoxyCodeLine{    if(T==NULL)\{}
\DoxyCodeLine{        return 0;}
\DoxyCodeLine{    \}}
\DoxyCodeLine{}
\DoxyCodeLine{    int count = 0;}
\DoxyCodeLine{}
\DoxyCodeLine{    count = count + (30 -\/ strlen(T-\/>book.titulo));}
\DoxyCodeLine{    count = count + (30 -\/ strlen(T-\/>book.primAutor));}
\DoxyCodeLine{    count = count + (30 -\/ strlen(T-\/>book.secAutor));}
\DoxyCodeLine{    count = count + (30 -\/ strlen(T-\/>book.editora));}
\DoxyCodeLine{    count = count + (30 -\/ strlen(T-\/>book.area));}
\DoxyCodeLine{    count = count + (30 -\/ strlen(T-\/>book.idioma));}
\DoxyCodeLine{}
\DoxyCodeLine{    return wastedLoopTree(T-\/>left) + wastedLoopTree(T-\/>right) + count;}
\DoxyCodeLine{\}}
\DoxyCodeLine{}
\DoxyCodeLine{int wastedMemory()\{}
\DoxyCodeLine{    Lista P = mainLista;}
\DoxyCodeLine{    Tree T = mainTree;}
\DoxyCodeLine{}
\DoxyCodeLine{    int total, clientes = 0, livros = 0;}
\DoxyCodeLine{}
\DoxyCodeLine{}
\DoxyCodeLine{    while (P!=NULL)\{}
\DoxyCodeLine{        clientes = clientes + (30 -\/ strlen(P-\/>Cli.Nome));}
\DoxyCodeLine{        clientes = clientes + (30 -\/ strlen(P-\/>Cli.MinhaMorada.Casa));}
\DoxyCodeLine{        clientes = clientes + (30 -\/ strlen(P-\/>Cli.MinhaMorada.Cidade));}
\DoxyCodeLine{}
\DoxyCodeLine{        P = P-\/>Prox;}
\DoxyCodeLine{    \}}
\DoxyCodeLine{}
\DoxyCodeLine{    livros = wastedLoopTree(T);}
\DoxyCodeLine{    }
\DoxyCodeLine{    total = livros*sizeof(char) + clientes*sizeof(char);}
\DoxyCodeLine{    }
\DoxyCodeLine{    return total;}
\DoxyCodeLine{\}}
\end{DoxyCode}
 Explicação\+: Estas duas funções trabalham em conjunto para determinar a memória desperdiçada pelas estruturas de dados dos Livros e Clientes. Como o espaço reservado para cada String é a mesma, este calculo é simples de fazer. A primeira função calcula o espaço desperdiçado pela árvore e o segundo calcula o desperdicio dos Clientes e junta os dois valores e devolve a quantidade de bytes desperdiçados.
\item Algoritmo Número 6\+: 
\begin{DoxyCode}{0}
\DoxyCodeLine{Cliente bigSpender(int ano, int mes)\{}
\DoxyCodeLine{    Lista P = mainLista;}
\DoxyCodeLine{    ListaCompra C = NULL;}
\DoxyCodeLine{    Cliente bigTimeSpender;}
\DoxyCodeLine{    float shmoney = 0, aux = 0;}
\DoxyCodeLine{}
\DoxyCodeLine{    while (P!=NULL)\{}
\DoxyCodeLine{        C = P-\/>Cli.lista;}
\DoxyCodeLine{}
\DoxyCodeLine{        if(C!=NULL)\{}
\DoxyCodeLine{            while(C!=NULL)\{}
\DoxyCodeLine{                if(C-\/>dataVenda.ano == ano \&\& C-\/>dataVenda.mes == mes)\{}
\DoxyCodeLine{                    aux = aux + C-\/>precoTotal;}
\DoxyCodeLine{                \}}
\DoxyCodeLine{                }
\DoxyCodeLine{                C = C-\/>Prox;}
\DoxyCodeLine{            \}}
\DoxyCodeLine{            printf("\%s\(\backslash\)n", P-\/>Cli.Nome);}
\DoxyCodeLine{            if(aux > shmoney)\{}
\DoxyCodeLine{                shmoney = aux;}
\DoxyCodeLine{                bigTimeSpender = newCliente(0,0,"NULL",newMorada("NULL","NULL",0000,000));}
\DoxyCodeLine{                bigTimeSpender = newCliente(P-\/>Cli.NIF,P-\/>Cli.telefone,P-\/>Cli.Nome,P-\/>Cli.MinhaMorada);}
\DoxyCodeLine{            \}}
\DoxyCodeLine{        \}}
\DoxyCodeLine{        }
\DoxyCodeLine{        P = P-\/>Prox;}
\DoxyCodeLine{    \} }
\DoxyCodeLine{}
\DoxyCodeLine{    return bigTimeSpender;}
\DoxyCodeLine{\}}
\end{DoxyCode}
 Explicação\+: Esta função procura faz o somatório de valor gasto dos Clientes num dado mês e ano, guardando sempre o maior valor gasto e o \mbox{\hyperlink{structCliente}{Cliente}} em questão. No final devolve o \mbox{\hyperlink{structCliente}{Cliente}} que mais gastou nesse período de tempo.
\item Algoritmo Número 7\+: 
\begin{DoxyCode}{0}
\DoxyCodeLine{int numLivrosComprados(Cliente P)\{}
\DoxyCodeLine{    ListaCompra C = P.lista;}
\DoxyCodeLine{    int numL = 0;}
\DoxyCodeLine{}
\DoxyCodeLine{    while(C != NULL)\{}
\DoxyCodeLine{        numL = numL + C-\/>quantidade;}
\DoxyCodeLine{        C = C-\/>Prox;}
\DoxyCodeLine{    \}}
\DoxyCodeLine{    }
\DoxyCodeLine{    return numL;}
\DoxyCodeLine{\}}
\DoxyCodeLine{}
\DoxyCodeLine{Cliente maisLivrosComprados()\{}
\DoxyCodeLine{    Lista P = mainLista;}
\DoxyCodeLine{    Cliente cli;}
\DoxyCodeLine{    int aux = 0, final = 0;}
\DoxyCodeLine{}
\DoxyCodeLine{    while(P != NULL)\{}
\DoxyCodeLine{        aux = numLivrosComprados(P-\/>Cli);}
\DoxyCodeLine{        if(final < aux)\{}
\DoxyCodeLine{            final = aux;}
\DoxyCodeLine{            cli = P-\/>Cli;}
\DoxyCodeLine{        \}}
\DoxyCodeLine{}
\DoxyCodeLine{        P = P-\/>Prox;}
\DoxyCodeLine{    \}}
\DoxyCodeLine{    return cli;}
\DoxyCodeLine{\}}
\end{DoxyCode}
 Explicação\+: Estas duas funções trabalham juntas para determinar que \mbox{\hyperlink{structCliente}{Cliente}} mais Livros comprou. A primeira determina o numero de Livros comprados por um \mbox{\hyperlink{structCliente}{Cliente}} expecifico. A segunda é o loop pela Lista de Clientes invocando a primeira função, guardando sempre o maior numero de livros e os eu respetivo \mbox{\hyperlink{structCliente}{Cliente}}. No final devolve o \mbox{\hyperlink{structCliente}{Cliente}} com mais Livros comprados.
\item Algoritmo Número 8\+: 
\begin{DoxyCode}{0}
\DoxyCodeLine{int contagemPorAno(Tree T, int ano)\{}
\DoxyCodeLine{    if(T==NULL)\{}
\DoxyCodeLine{        return 0;}
\DoxyCodeLine{    \}}
\DoxyCodeLine{}
\DoxyCodeLine{    if(T-\/>book.anoPub == ano)\{}
\DoxyCodeLine{        return contagemPorAno(T-\/>left,ano)+contagemPorAno(T-\/>right,ano)+1;}
\DoxyCodeLine{    \}}
\DoxyCodeLine{}
\DoxyCodeLine{    return contagemPorAno(T-\/>left,ano)+contagemPorAno(T-\/>right,ano);}
\DoxyCodeLine{\}}
\DoxyCodeLine{}
\DoxyCodeLine{int anoMaisPublicacoes()\{}
\DoxyCodeLine{    int aux = 0, num, anoFinal, anoInicial = 2010;}
\DoxyCodeLine{}
\DoxyCodeLine{    while (anoInicial<=2021)\{}
\DoxyCodeLine{        num = contagemPorAno(mainTree,anoInicial);}
\DoxyCodeLine{}
\DoxyCodeLine{        if(num>aux)\{}
\DoxyCodeLine{            aux = num;}
\DoxyCodeLine{            anoFinal = anoInicial;}
\DoxyCodeLine{        \}}
\DoxyCodeLine{}
\DoxyCodeLine{        ++anoInicial;}
\DoxyCodeLine{    \}}
\DoxyCodeLine{}
\DoxyCodeLine{}
\DoxyCodeLine{    return anoFinal;}
\DoxyCodeLine{\}}
\end{DoxyCode}
 Explicação\+: Novamente duas funções que trabalham juntas para atingir um fim. A primeira faz a contagem de Livros num dado ano. A segunda vai verificar a quantidade de Livros para cada ano desde 2010 até 2021, devolvendo no final o ano com mais publicações. 
\end{DoxyEnumerate}